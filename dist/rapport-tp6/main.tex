\documentclass[a4paper,11pt]{article}
\usepackage[T1]{fontenc}
\usepackage[utf8]{inputenc}
\usepackage{lmodern}
\usepackage[francais]{babel}

% title info
\newcommand{\ititle}{IA04 - TP6} % title
\newcommand{\isubtitle}{RDF et Requêtes SPARQL}
\newcommand{\iauthor}{Marouane Hammi\\Adrien Jacquet} % author
\newcommand{\idate}{\today} % date

\input{src/config}
\input{src/source_code}

\begin{document}
\input{src/titlepage}

\newpage
\section{Requêtes sur une ontologie}
\begin{enumerate}
\item Pour rechercher l'ensemble des concepts définis dans une ontologie, on utilise la requête SPARQL suivante :
\begin{lstlisting}[language=SPARQL]
SELECT DISTINCT ?c WHERE {?c rdf:type rdfs:Class}
\end{lstlisting}
~\\\vspace*{-.5cm}
\item La recherche des propriétés d'une ontologie suit globalement le même schéma, sauf que l'on cherche non pas les éléments de type \lstinline$rdfs:Class$, mais les éléments de type \lstinline$rdf:Property$.
\begin{lstlisting}[language=SPARQL]
SELECT DISTINCT ?p WHERE {?p rdf:type rdf:Property}
\end{lstlisting}
~\\\vspace*{-.5cm}
\item Pour récupérer la liste des concepts étant le domaine d'au moins deux relations, on commence par sélectionner, comme précédemment, la liste des concepts de l'ontologie.
On ne conserve ensuite que les concepts qui participent à au moins deux triplets respectant le motif \lstinline$?x rdfs:domain ?concept$ différents.
\begin{lstlisting}[language=SPARQL]
SELECT DISTINCT ?c WHERE {
	?c rdf:type rdfs:Class .
	?x rdfs:domain ?c .
	?y rdfs:domain ?c .
	FILTER (?x != ?y)
}
\end{lstlisting}
\end{enumerate}

\newpage
\section{Système multi-agent}
\subsection{Requêtes sur la base de connaissance}
\begin{enumerate}
	\item Pour récupérer la liste des individus connus par une personne désignée par son nom, il faut procéder en deux étapes.
	On commence par retrouver l'identifiant RDF de l'entité ayant une propriété \lstinline$foaf:firstName$ valant \lstinline!$name!, c'est à dire le nom passé en paramètre pour la requête.
	On retrouve ensuite la liste des personnes connus en sélectionnant la liste des triplets dont le prédicat est \lstinline$foaf:knows$ et le sujet l'identifiant récupéré précédemment.
	\begin{lstlisting}[language=SPARQL]
SELECT DISTINCT ?p WHERE {
	?s foaf:firstName "$name" .
	?s foaf:knows ?p .
}
	\end{lstlisting}
	~\\\vspace*{-.5cm}
	\item Pour rechercher les personnes ayant un intérêt pour un même pays qu'un individu dont le nom est spécifié en paramètre, on doit commencer, comme précédemment par récupérer l'identifiant RDF de cet individu.
	On récupère ensuite les objets des triplets de la forme \lstinline$?s foaf:topic_interest ?c$ où \lstinline$?s$ est la personne retrouvée précédemment et \lstinline$?c$ une entité de type \lstinline$lgdo:Country$, c'est à dire étant impliqué dans un triplet \lstinline$?c a lgdo:Country$.
	Enfin, on sélectionne les triplets respectant encore une fois la forme \lstinline$?p foaf:topic_interest ?c$, mais où \lstinline$?p$ est différent de l'identifiant de la personne dont le nom a été passé en paramètre.
		\begin{lstlisting}[language=SPARQL]
SELECT DISTINCT ?p WHERE {
	?s foaf:firstName "$name" .
	?s foaf:topic_interest ?c .
  ?c a lgdo:Country .
  ?p foaf:topic_interest ?c .
  FILTER(?p != ?s)
}
	\end{lstlisting}
\end{enumerate}
\subsection{Description des agents}
Le système est composé de deux agents : PropagateSPARQLAgt et KBAgt.
\begin{itemize}
\item PropagateSPARQLAgt est l'agent chargé de l'interface du système avec l'utilisateur. Il possède les Behaviours suivants :
	\begin{itemize}
		\item InterfaceBhv : CyclicBehaviour qui attend un message REQUEST comportant le type de requête souhaitée et le nom (paramètre nécessaire pour certaines requêtes). Lorsqu'une requête est envoyée, un Behaviour est créé pour recevoir la réponse.
		\item ReceiveBhv : Behaviour créé pour attendre la réponse d'une requête contenue dans un message INFORM et afficher la réponse dans la sortie standard. Un conversationID est utilisé pour garantir que la réponse lui est bien destiné.
	\end{itemize}
\item KBAgt est l'agent chargé d'effectuer les requêtes à la base de connaissance.
	\begin{itemize}
		\item ReceiveBhv : CyclicBehaviour qui attend un message REQUEST comportant la requête souhaitée. Il concatène la requête avec la liste des préfixes et effectue la requête sur la base de connaissance. Une fois la réponse obtenue, elle est renvoyée à l'expéditeur de la requête dans un message INFORM.
	\end{itemize}
\end{itemize}

\section{Requêtes vers un site distant}
\subsection{Composition}
Pour récupérer la liste des capitales des pays qui intéressent une personne, il faudrait effectuer deux types différents de requête.
La première, sur la base de connaissance locale, pour récupérer la liste de ces pays :
\begin{lstlisting}[language=SPARQL]
SELECT DISTINCT ?c WHERE {
	?s foaf:firstName "$name" .
	?s foaf:topic_interest ?c .
  ?c a lgdo:Country .
}
\end{lstlisting}

La seconde, exécutée une fois par pays précédemment sélectionné sur la base de connaissance distante, pour récupérer la capitale associée :
\begin{lstlisting}[language=SPARQL]
SELECT ?city WHERE { 
  "$country" lgdo:capital_city ?city .
} 
\end{lstlisting}
\subsection{Description des agents}
Pour ce système nous utilisons le PropagateSPARQLAgt précédent, qui va selon le type de requête demandé, envoyer la requête à l'agent KBAgt ou à l'agent GeoAgt.
L'agent GeoAgt fonctionne de la même manière que KBAgt, excepté qu'il effectue la requête sur la base de connaissance distante.

Pour effectuer la requête composée, il faudrait que :
\begin{enumerate}
\item PropagateSPARQLAgt reçoive le type de la requête venant de l'utilisateur, dans son InterfaceBhv.
\item InterfaceBhv envoie la première requête à KBAgt et créé le ReceiveBhv en précisant qu'il s'agit d'une requête composée. 
\item Lorsque ReceiveBhv reçoit la réponse de KBAgt usuelle, il crée un ComposeBhv.
\item Le ComposeBhv envoie les requêtes au GeoAgt de la même manière que InterfaceBhv et crée des ReceiveBhv qui afficheront les différents résultats obtenus.
\end{enumerate}
 

\end{document}
